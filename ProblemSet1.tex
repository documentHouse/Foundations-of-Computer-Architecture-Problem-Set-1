%\documentclass[11pt,reqno]{amsart}
\documentclass[11pt,reqno]{article}
\usepackage[margin=.8in, paperwidth=8.5in, paperheight=11in]{geometry}
%\usepackage{geometry}                % See geometry.pdf to learn the layout options. There are lots.
%\geometry{letterpaper}                   % ... or a4paper or a5paper or ... 
%\geometry{landscape}                % Activate for for rotated page geometry
%\usepackage[parfill]{parskip}    % Activate to begin paragraphs with an empty line rather than an indent7
\usepackage{graphicx}
\usepackage{pstricks}
\usepackage{amssymb}
\usepackage{epstopdf}
\usepackage{amsmath}
\usepackage{subfigure}
\usepackage{caption}
\pagestyle{plain}
%\renewcommand{\topfraction}{0.3}
%\renewcommand{\bottomfraction}{0.8}
%\renewcommand{\textfraction}{0.07}
\DeclareGraphicsRule{.tif}{png}{.png}{`convert #1 `dirname #1`/`basename #1 .tif`.png}

\title{Foundations of Computer Architecture: \\ Problem Set 1 }
\author{Andrew Rickert}
\date{Professor: Dr. Horace Malcolm \\ \hspace{-25pt} Due Date: February 6,  2012}                                           % Activate to display a given date or no date

\begin{document}
\maketitle


% Page 1
\begin{flushleft} 
Problem 1 \\
\rule{500pt}{1pt}\\
\end{flushleft} 
\framebox{Part a}\\ 

We let $T_B$ and $T_A$ be the execution time before and after the speed up respectively. Similarly we let $CPI^B_M$ and $CPI^A_M$ be the $CPI$ of the multiply instruction before and after the speed up respectively. We will use $CPI_O$ to denote the average $CPI$ of the other instructions. \\
If $I_T$ is the total number of instructions, and $C_R$ is the clock rate then the expressions for $T_A$ and $T_B$ are:
\begin{eqnarray*}
T_A &=& \frac{CPI^A_M \cdot(0.22 I_T) + CPI_O \cdot (0.78 I_T)}{C_R} \\
T_B &=& \frac{CPI^B_M \cdot(0.22 I_T) + CPI_O \cdot (0.78 I_T)}{C_R}
\end{eqnarray*}

The speedup is defined as the ratio of the execution time before the speed up to after the speed up. The following calculation gives an expression for the speed up:
\begin{eqnarray*}
\text{Speed up} = \frac{T_B}{T_A} &=&  \frac{\frac{CPI^B_M \cdot(0.22 I_T) + CPI_O \cdot (0.78 I_T)}{C_R}}{\frac{CPI^A_M \cdot(0.22 I_T) + CPI_O \cdot (0.78 I_T)}{C_R}} \\
&=& \frac{CPI^B_M \cdot(0.22 I_T) + CPI_O \cdot (0.78 I_T)}{CPI^A_M \cdot(0.22 I_T) + CPI_O \cdot (0.78 I_T)} \\
&=& \frac{CPI^B_M \cdot 0.22 + CPI_O \cdot 0.78}{CPI^A_M \cdot 0.22 + CPI_O \cdot 0.78}
\end{eqnarray*} \\

So the equation for the speed up is 
\begin{equation}
\text{Speed Up} =  \frac{CPI^B_M \cdot 0.22 + CPI_O \cdot 0.78}{CPI^A_M \cdot 0.22 + CPI_O \cdot 0.78} \label{eqn:speedup}
\end{equation}

Using the values given in the problem we have 
\[ \text{Speed Up} =  \frac{CPI^B_M \cdot 0.22 + CPI_O \cdot 0.78}{CPI^A_M \cdot 0.22 + CPI_O \cdot 0.78} =  \frac{4 \cdot 0.22 + 4 \cdot 0.78}{3 \cdot 0.22 + 4 \cdot 0.78} = 1.06 \]

\newpage
\noindent\framebox{Part b}\\ 

If we solve equation in $(\ref{eqn:speedup})$ for $CPI^B_M$ in terms of the speed up then we have the $CPI$ before the speed up in terms of the given value for the speed up. This will solve the problem. Performing the required algebra on equation $(\ref{eqn:speedup})$ gives the following:
\begin{eqnarray*}
CPI^B_M &=& \frac{1}{0.22} \text{SpeedUp} \cdot (0.22 \cdot CPI^A_M + 0.78 \cdot CPI_O) - \frac{1}{0.22} 0.78 \cdot CPI_O \\
		&=& \frac{1}{0.22} \text{1.76} \cdot (0.22 \cdot 3 + 0.78 \cdot 4) - \frac{1}{0.22} 0.78 \cdot 4\\
		&\backsimeq& 16
\end{eqnarray*}



\begin{flushleft} 
Problem 2 \\
\rule{500pt}{1pt}\\
\end{flushleft} 
\framebox{Part a}\\ 

The clock rate, $C_R$ of the processor, is the inverse of cycle time so 
\[ C_R = \frac{1}{\text{Cycle Time}} = \frac{1}{50 \cdot 10^{-12}} = \frac{1000 \cdot 10^9}{50} = 20 \cdot 10^9, \; \text{which is  20 GHz}\]
We would also like to calculate the total number of clock cycles for the program. We let $CPI_D$ and $CPI_O$ be the $CPI$ of the division instruction and the other instructions respectively. If $I_T$ is the total number of instructions then the total clock cycles $CC_T$ is 
\begin{eqnarray} 
CC_T &=& CPI_D \cdot 0.1 \cdot I_T +  CPI_O \cdot 0.9 \cdot I_T  \label{eqn:totalclocks} \\
	   &=& 12 \cdot 0.1 \cdot 20 \cdot 10^6 + 4 \cdot 0.9 \cdot 20 \cdot 10^6 = 96 \cdot 10^6 \; \text{instructions} \nonumber
\end{eqnarray}

\noindent \framebox{Part b}\\ 

We will need to find the number of clock cycles before $CC_T^B$ and after $CC_T^A$ the speed up. The value for before the speed up was calculated in the part a of this problem as $CC_T^B = 96 \cdot 10^6$ instructions. To calculate the number of clock cycles after the speed up we use equation $(\ref{eqn:totalclocks})$ again to get
\begin{eqnarray*} 
CC_T^A &=& CPI_D \cdot 0.1 \cdot I_T +  CPI_O \cdot 0.9 \cdot I_T  \\
	   &=& 2 \cdot 0.1 \cdot 20 \cdot 10^6 + 4 \cdot 0.9 \cdot 20 \cdot 10^6 = 76 \cdot 10^6 \; \text{instructions}
\end{eqnarray*}

If $C_R$ is the clock rate, and $T_B$,$T_A$ are the execution times before and after the speed up then the speed up can be calculated:
\[ \text{Speed Up} =  \frac{T_B}{T_A} = \frac{\frac{CC_T^B}{C_R}}{\frac{CC_T^A}{C_R}} = \frac{CC_T^B}{CC_T^A} = \frac{96 \cdot 10^6}{76 \cdot 10^6} = 1.26\]

\noindent \framebox{Part c}\\ 

Writing a bunch of stuff and then moving on to the next part..\\

\noindent \framebox{Part d}\\ 

Ok, now it is time to write some more stuff.. \\


\begin{flushleft} 
Problem 3 \\
\rule{500pt}{1pt}\\
\end{flushleft} 
\framebox{Part a}\\ 

Writing a bunch of stuff and then moving on to the next part..\\

\noindent\framebox{Part b}\\ 

Ok, now it is time to write some more stuff.. \\

\noindent \framebox{Part c}\\ 

Writing a bunch of stuff and then moving on to the next part..\\

\noindent \framebox{Part d}\\ 

Ok, now it is time to write some more stuff.. \\

\end{document}  